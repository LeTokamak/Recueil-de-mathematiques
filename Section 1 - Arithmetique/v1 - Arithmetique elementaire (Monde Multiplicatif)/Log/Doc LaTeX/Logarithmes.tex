\documentclass[a4paper]{article}

\usepackage[T1]{fontenc}
\usepackage[utf8x]{inputenc}
%\usepackage[mathletters]{ucs}
%\usepackage[french]{babel}

\DeclareFontEncoding{LS1}{}{}
\DeclareFontSubstitution{LS1}{stix}{m}{n}
\DeclareSymbolFont{symbols4}{LS1}{stixbb}{m}{it}
\DeclareMathSymbol{\varhexagonblack}{\mathord}{symbols4}{"DD}
\DeclareMathSymbol{\hexagonblack}   {\mathord}{symbols4}{"DE}

\usepackage{amssymb}
\def\nbR{\ensuremath{\mathrm{I\! R}}}


\parindent=1cm

\begin{document}

	\begin{titlepage}
		\begin{center}
		
\Huge  		\textbf{Mathématiques}\\
		\bigskip \smallskip
		
\Large		\textbf{$\varhexagonblack$ Logarithmes $\varhexagonblack$}\\
		\bigskip
		
\large		Clément \sc{Campana}  \\ 
		\smallskip 
		
\normalfont	Octobre 2021\\
		\bigskip \bigskip \bigskip
		
		\end{center}
Sommaire\\

	\end{titlepage}

\section*{Origine des Logarithmes}

	Les logarithmes ont d'abord été utilisé pour transformer les multiplications et les divisions qui sont difficiles à calculer, en additions et soustraction. Avec l'arrivée des ordinateurs, ils ont perdus cette utilité...\\
	
	Mais depuis leur création en 1614 par John Napier (ou Neper), ils ont acquis beaucoup d'autres utilités en science.\\
	
	%% Explication du passage du monde des addition au monde des multiplications

\pagebreak

\section*{Types de Logarithme}

	Il existe plusieurs types de logarithme en fonction de la \textbf{base} dans laquelle ils sont exprimés, ils sont notés :
	
	{\huge $$ \log_{a}(x) $$}
	\begin{center}
	\textit{Logarithme de x exprimé dans la base a.}\\
	\end{center}
	
\bigskip	
	
Il est définit de manière à respecter cette égalité :
	{\LARGE $$ a ^{ \log_{a} (x)}= x $$\\}

\subsection*{Logarithme décimal}

	Le logarithme en base 10 est appelé \textbf{logarithme décimale}, et cette base est très pratique car c'est la même que celle dans laquelle on compte, grâce à ça on a des propriétés intéressantes :	
	
	\begin{center}
\begin{Large}
	\begin{tabular}{c|c|c}
	$ \log_{10} (10) = 1 $ &
	$ \log_{10} (100) = 2 $ &
	$ \log_{10} (10^{n}) = n $ \\
	\end{tabular}
\end{Large}
	\end{center}

	Par simplification, on préfèrera noter le logarithme décimal :	{\Large $ \log (x) $}\\
	
	C'est dans cette base là que les logarithmes de la table sont exprimés, car leurs propriétés que nous verrons plus loin simplifie beaucoup les gros calculs.\\

\subsection*{Logarithme népérien}

	Le logarithme népérien est exprimé en base \textbf{e}. Ce \textbf{e} est un nombre irrationnel (il n'a pas de valeur exact) mais celui-ci occupe une place très importante dans les mathématiques.  Il a pour certain la même place que le nombre $ \pi $.\\
	
	Ce logarithme est considéré comme étant celui le plus important, car on le retrouve dans beaucoup d'équation. Mais on préfère le remplacer par le logarithme décimale ($ \log (x) $) qui est plus intuitif à utiliser.\\
	
	On note ce logarithme : {\Large $ \ln (x) $}\\
	
	De plus, à partir de $ \ln (x) $, on peut retrouver tous les logarithmes des autres bases grâce à cette égalité : $$ \log_a (x) = \frac{\ln (x)}{\ln (a)} $$

\pagebreak

\section*{Propriétés des Logarithmes}

	{\Huge $$ \log_a ( x \times y ) = \log_a (x) + \log_a (y) $$}
	
	{\large $$ \log_a \left( \frac{x}{y} \right)       = \log_a (x) - \log_a (y) $$ }

\begin{center}
\begin{large}
	\begin{tabular}{c|c}
	$ \log_a ( x^n )               =   n \log_a (x) $ &
	$ \log_a \left( \frac{1}{x^{n}} \right) = - n \log_a (x) $ \\
	\end{tabular}
\end{large}
	\end{center}
	
	Ces propriétés (et surtout la première) sont extrêmement utiles pour simplifier des multiplications et des divisions, car elles transforment celle-ci en simple addition et soustraction !\\
	
	\textit{cf. Démonstration en Annexe}
	
	
	
	

\pagebreak

\section*{Changement de base de Logarithme}

\pagebreak

\section*{Lecture de la table}

	Cette partie détaille les critères de divisibilité des nombres, inférieurs à 100 qui sont premiers ou qui sont des puissances de nombre premier.\\

	\par On rappel que : $ \forall n \in \mathbb{N}: n = p_{1}^{\lambda_1} \times \ldots \times p_{r}^{\lambda_r}$ avec $\{p_1,\cdots, p_r\}$ des nombres premiers et $\{\lambda_1,\cdots, \lambda_r\}$ des entiers.\\
	Par exemple : $12 = 3^1 \times 4^1$ donc un nombre est divisible par $12$ si et seulement s'il est divisible par $3$ et $4$.\\

	\par On représentera un entier naturel de $k + 1$ chiffres par $\overline{a_k \cdots a_1 a_0}$, où $a_0$ est le chiffre des unités, $a_1$ des dizaines, $a_2$ des centaines, etc.\\

	\par On appellera $n$ le nombre dont on cherche à trouver les diviseurs.\\

	\par $n$ est divisible par $2 \Leftrightarrow 2$ divise $n \Leftrightarrow$ \Large $^2|_n$

	\bigskip \bigskip \bigskip

\subsection{Critères de divisibilité des nombres jusqu'à 10}

	\huge
	\bigskip

	\begin{tabular}{l|l}


		$^2|_n \Leftrightarrow$ $a_0 \in \{ 0,2,4,6,8 \}$ &
		$^3|_n \Leftrightarrow$ $^3|_{a_0 + \ldots + a_k}$ \\
	
	\tabularnewline

		$^4|_n \Leftrightarrow$ $^4|_{2a_1 + a_0}$ &
		$^5|_n \Leftrightarrow$ $a_0 \in \{ 0,5 \}$ \\
	
	\tabularnewline	
	
		$^6|_n \Leftrightarrow$ $^2|_n$ et $^3|_n$ &
		$^7|_n \Leftrightarrow$ $^7|_{\overline{a_k \cdots a_2 a_1} - 2 a_0}$ \\
	
	\tabularnewline	
	
		$^8|_n \Leftrightarrow$ $^2|_{4a_2 + 2a_1 + a_0}$ &
		$^9|_n \Leftrightarrow$ $^9|_{a_0 + \ldots + a_k}$ \\

	\end{tabular}

\pagebreak
\normalsize
\subsection{Critères de divisibilité des puissances de 2,3 et 5}

$\alpha \in \mathbb{N}$

\subsubsection*{Critère de divisibilité par $2^\alpha$}

$^{2^{\alpha}}|_n \Leftrightarrow$ $a_0 \in \{ 0,2,4,6,8 \}$

	Un nombre est divisible par 2n si et seulement si ses n derniers chiffres forment un nombre divisible par 2n.

Exemples
	Un nombre est divisible par 16 = 24 si et seulement si le nombre formé par ses 4 derniers chiffres est divisible par 16.
	Un nombre est divisible par 32 = 25 si et seulement si le nombre formé par ses 5 derniers chiffres est divisible par 32. 
Par exemple : 87 753 216 864 est divisible par 32 car 16 864 est divisible par 32.


\subsubsection*{Critère de divisibilité par $3^n$}

	On regroupe les chiffres d'un nombre en partant de la droite n par n.
	Le nombre est alors divisible par 3n si la somme de ces groupes est divisible par 3n.

Exemples
	2079108 est divisible par 33 = 27 car  		2 + 079 + 108 = 189 	= 7 × 27
	2079108 est aussi divisible par 34 = 81 car		207 + 9108 = 9315 	= 115 × 81








\subsubsection*{Critère de divisibilité par $5^n$}

	Un nombre est divisible par 5n si et seulement si ses n derniers chiffres forment un nombre divisible par 5n.


Exemples
	Un nombre est divisible par 25 = 52 si et seulement si le nombre formé par ses deux derniers chiffres est divisible par 25, c'est-à-dire si son écriture se termine par 00, 25, 50 ou 75.

Par exemple : 258 975 est divisible par 25 car il se termine par 75.
		 257 543 625 est divisible par 53 = 125 car 625 est divisible par 125.


\pagebreak

\section{Plus Grand Diviseur Commun (PGCD)}

\pagebreak

\section{Théorème de Bézout}

\pagebreak

\section{Théorème de Gauss}

\pagebreak

\end{document}